\section{Algorithm: Two Sum}

\subsection{Latex Source}

In this section, we present a simple algorithm to solve the "Two Sum" problem, where the goal is to find two numbers in an array that add up to a given target.

\begin{algorithm}
\caption{Two Sum}
\begin{algorithmic}[1]
\REQUIRE Array of integers \texttt{nums}, integer \texttt{target}
\ENSURE Indices of the two numbers such that they add up to \texttt{target}
\STATE Initialize an empty dictionary \texttt{num\_to\_index}
\FOR{each index $i$ and number \texttt{num} in \texttt{nums}}
    \STATE \texttt{complement} $\leftarrow$ \texttt{target} $-$ \texttt{num}
    \IF{\texttt{complement} is in \texttt{num\_to\_index}}
        \RETURN [\texttt{num\_to\_index[complement]}, $i$]
    \ENDIF
    \STATE \texttt{num\_to\_index[num]} $\leftarrow i$
\ENDFOR
\RETURN []
\end{algorithmic}
\end{algorithm}

The algorithm uses a hash map to store the indices of the numbers, allowing for efficient lookup of the complement needed to reach the target sum.


\subsection{Imported from File}

In this section, we present a simple algorithm to solve the "Two Sum" problem, where the goal is to find two numbers in an array that add up to a given target.

\lstinputlisting{files/demo_insert_script.py}

The algorithm uses a hash map to store the indices of the numbers, allowing for efficient lookup of the complement needed to reach the target sum.

